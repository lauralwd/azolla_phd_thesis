% Table of contents formatting
\renewcommand{\contentsname}{Table of Contents}
\setcounter{tocdepth}{0}

% use line numbers. Start and stop these per chapter.
\usepackage{lineno}

% Headers and page numbering
\usepackage{fancyhdr}

% use the minitoc package for a table of contents at the beginning of each chapter
\usepackage{minitoc}
\mtcsettitle{minitoc}{Chapter Contents}
\renewcommand{\mtcskip}{\vskip 0.2cm}
% Remove the top and bottom horizontal lines of the minitoc
\mtcsetrules{minitoc}{off}
% set minitoc depth to 1
\mtcsetdepth{minitoc}{1}
\mtcsettitlefont{minitoc}{\Large}

% hack disable chapters starting at the right side but have them start either left or right.
\makeatletter
\renewcommand{\chapter}{%
  \clearpage
  \thispagestyle{plain}%
  \global\@topnum\z@
  \@afterindentfalse
  \secdef\@chapter\@schapter}
\makeatother


% plain is used on chapter title pages and the thesis title pages unless overidden with \thispagestyle{somepagestyle}
\fancypagestyle{plain}{
  \fancyhead{}                       % clear headers
  \fancyfoot{}                       % clear footers
  \renewcommand{\headrulewidth}{0pt} % no line at top of page
}

\fancypagestyle{toc}{
  \fancyhead{}                       % clear headers
  \fancyfoot{}                       % clear footers
  \renewcommand{\headrulewidth}{0pt} % no line at top of page
  \fancyfoot[C]{~\thepage~}          % page numbers in Centre
}

\fancypagestyle{chapter}{
  \fancyhead{}                       % clear headers
  \fancyfoot{}                       % clear footers
  \renewcommand{\headrulewidth}{0pt} % no line at top of page
  \fancyfoot[RO,LE]{~\thepage~}      % page numbers Right Odd, Left Even
}

% % Following package is used to add background image to front page
% %\usepackage{wallpaper}

% Table package
\usepackage{ctable}% http://ctan.org/pkg/ctable

% Deal with 'LaTeX Error: Too many unprocessed floats.'
\usepackage{morefloats}
% or use \extrafloats{100}
% add some \clearpage

% % Chapter header
% \usepackage{titlesec, blindtext, color}
% \definecolor{gray75}{gray}{0.75}
% \newcommand{\hsp}{\hspace{20pt}}
% \titleformat{\chapter}[hang]{\Huge\bfseries}{\thechapter\hsp\textcolor{gray75}{|}\hsp}{0pt}{\Huge\bfseries}

% % Fonts and typesetting
% \setmainfont[Scale=1.1]{Helvetica}
% \setsansfont[Scale=1.1]{Verdana}
\setmainfont[Scale=1.0]{Arial}
\setsansfont[Scale=1.0]{Arial}

% FONTS
\usepackage{xunicode}
\usepackage{xltxtra}
\defaultfontfeatures{Mapping=tex-text} % converts LaTeX specials (``quotes'' --- dashes etc.) to unicode
% \setromanfont[Scale=1.01,Ligatures={Common},Numbers={OldStyle}]{Palatino}
% \setromanfont[Scale=1.01,Ligatures={Common},Numbers={OldStyle}]{Adobe Caslon Pro}
%Following line controls size of code chunks
% \setmonofont[Scale=0.9]{Monaco}
%Following line controls size of figure legends
% \setsansfont[Scale=0.8]{Optima Regular}

% CODE BLOCKS
\usepackage[utf8]{inputenc}
\usepackage{listings}
\usepackage{color}

%Attempt to set math size
%First size must match the text size in the document or command will not work
%\DeclareMathSizes{display size}{text size}{script size}{scriptscript size}.
%\DeclareMathSizes{12}{13}{7}{7}

% ---- CUSTOM AMPERSAND
% \newcommand{\amper}{{\fontspec[Scale=.95]{Adobe Caslon Pro}\selectfont\itshape\&}}

% HEADINGS
\usepackage{sectsty}
\usepackage[normalem]{ulem}
\sectionfont{\rmfamily\mdseries\Large}
\subsectionfont{\rmfamily\mdseries\scshape\large}
\subsubsectionfont{\rmfamily\bfseries\upshape\normalsize}
%\subsectionfont{\rmfamily\mdseries\scshape\normalsize}
%\subsubsectionfont{\rmfamily\bfseries\upshape\normalsize}

% Set figure legends and captions to be smaller sized sans serif font
\usepackage[font={footnotesize,sf}]{caption}

\usepackage{siunitx}

% Adjust spacing between lines
\usepackage{setspace}
% \singlespacing
\onehalfspacing
% \doublespacing
\raggedbottom

% Set margins and paper size
\usepackage[paperheight=24cm,paperwidth=17cm,top=2cm,bottom=2cm,left=1.5cm,right=1.5cm,bindingoffset=1cm]{geometry}

% Add space between pararaphs
% http://texblog.org/2012/11/07/correctly-typesetting-paragraphs-in-latex/
\setlength\parindent{0.5in}
\setlength{\parskip}{9pt}
\usepackage{indentfirst}
% \setlength{\parskip}{\baselineskip}

% Set colour of links to black so that they don't show up when printed
\usepackage{hyperref}
\hypersetup{colorlinks=false, linkcolor=black}

% Thumb index
\usepackage[height=auto,minheight=30pt]{thumbs}

% Tables
\usepackage{booktabs}
\usepackage{threeparttable}
\usepackage{array}
\usepackage{makecell}
\newcolumntype{x}[1]{%
>{\centering\arraybackslash}m{#1}}%

% use wraptables:
\usepackage{wrapfig}

% Allow for long captions and float captions on opposite page of figures
% \usepackage[rightFloats, CaptionBefore]{fltpage}

% Don't let floats cross subsections
% \usepackage[section,subsection]{extraplaceins}

% Rotate images and tables
\usepackage{float}
\usepackage{pdfpages}
\usepackage{pdflscape}
\usepackage{graphicx}
\usepackage{rotating}

% Custom math
\usepackage{bbold}
\DeclareMathOperator*{\argmin}{\arg\!\min}

% For use of \cref and \Cref used by pandoc secnos
\usepackage{cleveref}
